\setcounter{Sec}{0}
\setcounter{Step}{0}
\renewcommand{\subprocid}{{\procid}-01}
\renewcommand{\procname}	{Electrical Verifications and Instruments Initializations}
\newcommand{\procnameOne}	{\procid-01\ Electrical Verifications and Instruments Initializations}

\subsection{\subprocid\ \procname.}\label{sec:electrical}

\begin{table}[H]
	\centering
	\footnotesize
	\begin{tabular}{|L{3.0cm}|L{13.0cm}|}\hline
		\textbf{Task ID} & \subprocid{} \\\hline
		\textbf{Task name} & \procname{} \\\hline
		\textbf{Task description} & This task includes:
		\begin{minipage}[t]{\linewidth}
			\begin{itemize}[nosep,after=\strut]
				\item Verification of grounding of all racks and AC power sockets to use.
				\item Verification of the facilities AC supply voltages.
				\item RF TestBed deployment.
				\item Preparation of PXA.
				\item Preparation of oscilloscope.
				\item Connection of PXI, RF TestBed and PXA to the \fmr{} network.
				\item RDP Connections from thin clients. See table \ref{tb:rdp}. 
			\end{itemize}
		\end{minipage} \\
		\hline
		\textbf{Task purpose} & Prepare \comEgse, \fmr{} and instruments for the execution of 
		aliveness, functional and performace tests of the communication system.\\
		\hline
		\textbf{Success criteria} & All electrical verifications are correct. Instruments powered-on and ready to perform measurements.\\
		\hline
		\textbf{Test Setup} & -\\
		\hline
		\textbf{Duration} & 60 minutes.\\
		\hline
		\textbf{Data sets required} & -\\
		\hline
		\textbf{Prerequisites} & 
		\begin{minipage}[t]{\linewidth}
			\begin{itemize}[nosep,after=\strut] 
				\item GS-GSE-FM (R) initialized according to GS-GSE test procedures (\refAD{fm-ver-delta-pro}) or  user manual (\refRD{gs-gse-fm}). % GS-GSE test procedures (\refAD{fm-ver-delta-pro}) or 
				\item GS-GSE-FM (R) configured according to Control Configuration Document (\refAD{gse-cfg-r}) and with their modified RF interfaces 
				(\refAD{reportFMRMod}).
				\item \comEgse{}{} initialized according to \comEgse{}{} user manual (\refAD{cegse}).
				\item RF TestBed powered off and only connected to facilities GND.
				\item Hardware: The necessary items are shown in the table \ref{tab:req-items} 
			\end{itemize} 
		\end{minipage}\\
		\hline
	\end{tabular}
	\caption{Procedure \subprocid{} \ description. } \label{tb:electrical}
\end{table}

\begin{table}[H]
	\centering
	\scriptsize
	\begin{tabular}{|L{2.5cm}|L{2.5cm}|L{2.5cm}|L{2cm}|L{2cm}|}
		\hline
		\textbf{Name}                   & \textbf{OW used}             & \textbf{IP}             & \textbf{User}          & \textbf{Password}  \\ \hline
		\comEgse{}             & OW Data A  & \comEgseIP{}   & EGSE COM      & Conae1234 \\ \hline%DUT MandC
		%TestBed-Cortex CRT-Q-D & OW TMTC A  & \crtTBIP{}     & cortex        & cortex    \\ \hline%S-Band Noise generation
		TestBed-Cortex HDR-XXL & OW TMTC A  & \hdrTBIP{}     & cortex        & cortex    \\ \hline%X-Band Noise generation
		GS-GSE.WIN8            & OW TMTC A  & \vmWinIP{}     & admin         & Sb1.C0n43 \\ \hline%SBMA and XBMA
		%TM\&TC Modem           & OW TMTC A  & \crtIP{}       & cortex        & cortex    \\ \hline%TM downlink
		Data Demodulator       & OW TMTC A  & \hdrIP{}       & cortex        & cortex    \\ \hline%Data Downlink 
		GS-GSE.MGMT            & OW TMTC A  & \vmTestingIP{} & administrator & Sb1.C0n43 \\ \hline%GS flows execution
		%GS-GSE.WIN8            & OW TMTC A  & \vmWinIP{}     & admin         & Sb1.C0n43 \\ \hline%SBMA
		%TM\&TC Modem           & OW TMTC A  & \crtIP{}       & cortex        & cortex    \\ \hline%TC uplink 
	\end{tabular}
	\caption{Initial RDP connections.}
	\label{tb:rdp} 
\end{table}
\newpage

\report{}{}{}

\begin{stepstable}{\subprocid{} \procname{}}
	\ExecutorRecord
	\SessionIdRecord
	%%%%%%%%%%%%%%%%%%%%%%%%%%%%%%%%%%%%%%%%%%%%%%%%%%%%%%%%%%%%%
	% ELECTRICAL VERIFICATIONS
	\SectionHeaderWithReset{Grounding and AC power verification}
	%%%%%%%%%%%%%%%%%%%%%%%%%%%%%%%%%%%%%%%%%%%%%%%%%%%%%%%%%%%%%
	\CheckPowerSupplyVoltageOn{Instrumentation bench}{}{}
    \InstallAndVerifyGroundOfPXA
	\InstallAndVerifyGroundOfOscilloscope
	%\PowerOnOscilloscope

       
	\VerifyGroundCEGSE{}
	    \VerifyGroundTB
	%\VerifyGroundAux{}
	%\VerifyGround{\rckttBB}{\rckttRF}{N}
	%\VerifyGround{\rckdtBB}{\rckttRF}{N}
	\VerifyGround{\rckttBB}{\rckttRF}{R}
	\VerifyGround{\rckdtBB}{\rckttRF}{R}
	\CheckPowerSupplyVoltageOn{Rack \comEgse}{}{}
    \CheckPowerSupplyVoltageOn{\rackRfTestBed}{}{}
	%\CheckPowerSupplyVoltageOn{\rckaux}{}{}
	%\CheckPowerSupplyVoltageOn{\rckttRF}{\rckttBB}{ of \textbf{\gse-FM (N)}}
	%\CheckPowerSupplyVoltageOn{\rckdtRF}{\rckdtBB}{ of \textbf{\gse-FM (N)}}
	\CheckPowerSupplyVoltageOn{\rckttRF}{\rckttBB}{ of \textbf{\gse-FM (R)}}
	\CheckPowerSupplyVoltageOn{\rckdtRF}{\rckdtBB}{ of \textbf{\gse-FM (R)}}

    \SectionHeaderWithReset{Installation of the Instruments}
		\ConnectExternalReferenceSignalToPXI
		\ConnectDcBlockToPxi
		\PowerOnPXA
		\ConnectPxaExternalFreqReferenceTo{the Power Splitter Data of \fmr{}}{REF1.01}{PXA}
		\StartVSASoftware
		\VerifyVsaRfInput
		\ConfigurePxaAsSpectrumAnalyzer
		\ConnectDcBlockToPxa
		\PowerOnOscilloscope
    
	\SectionHeaderWithReset{RF TestBed deploy }
		%TODO: preguntar a juan cual verificacion de las siguientes se hace
    	%\VerifyIfAndRfConnectiosOfDataTestBed
		%\VerifyIfAndRfConnectiosOfDataTestBedOneChannelTest
		\VerifyIfAndRfConnectiosOfDataTestBedOneChannelTest

		\ConnectTestBedToPowerSupply
		\TurnOnPDUTestBed
		\TurnOnDataTestBedComponents
    
	\SectionHeaderWithReset{Network connections}
		\ConnectCegseToGseNetwork
		\ConnectPxaToGseNetwork
		\ConnectTestBedToGseNetwork{\fmr{}}

	\SectionHeaderWithReset{Remote Connections from Thin Client}
		\RDPToPXIFromThinClient{\thinDT A}
		\RDPToMgmtFromThinClient{\thinTT A}
		\RdpToCortexHdrFromThinClientTTA
		\RdpToCortexHdrTBFromThinClientTTA
		\RDPToSbaVmFromThinClientA

	\SectionHeaderWithReset{CEGSE NTP Client}
		\ConfigureNtpClientInCegse
	
\end{stepstable}
\begin{longtable}{|p{17.0cm}|}
\endfirsthead
\endfoot
\caption{Procedure \subprocid \ table.} 

\end{longtable}