\renewcommand{\procid}		{SB1FS-COM-P-013}
\renewcommand{\subprocid}	{SB1FS-COM-P-013-06}
\renewcommand{\procname}	{\TestPerfFilterVector}

\newpage
\subsection{SB1FS-COM-P-013-06 \TestPerfFilterVector}
\label{sec:data-filter}

\renewcommand{\foldername}{cortex-screenshot-013-06}

\begin{table}[H] 
	\centering
	\footnotesize
	\begin{tabular}{|L{3.0cm}|L{13.0cm}|}\hline
		\textbf{Task ID}            & \subprocid{}\\\hline
		\textbf{Task name}          & \procname{}\\\hline
	 	\textbf{Task description} & 
In this test the EWC30 TX is set in Modulation mode. 
The modulated signal is received through the N1 [X-Band] 
interface of the \gse-FM (R) in the case of the \textbf{EWC30-FM1} 
and through N2 in the case of the \textbf{EWC30-FM2}. Two filter configurations 
in Data Demodulator (Cortex HDR) are evaluated 
(see table \ref{tab:data-advanced}).
A vector analysis of the received signals is carried 
out using the VSA and the Vector Script.
\\\hline
		\textbf{Task purpose} & The purpose of this test is to evaluate the two filters 
		configurations (see table \ref{tab:data-advanced}) in the
		Cortex HDR.% configuration returns the most accurate values: the minimum EVM value, minimum
		%Unbalanced ratio (or Amplitude Error) and minimum Phase error. 
		 On the other 
		hand, it is to perform a vector analysis of the modulated signals. \\
		\hline
		\textbf{Success criteria} & 
		
		\begin{itemize}[nosep,after=\strut]
			\item The  \textbf{A10} and \textbf{B2} filter configurations are 
		evaluated. %and one is identified 
		%as the one with the best performance. 
		\item Vector analysis is performed.
		\item For \textbf{A10}  filter configuration
		\begin{itemize}
			\item EVM < 6\%
			\item |Amplitude Error| < 0.5 dB rms  
			\item |Phase Error| < 5° rms for \textbf{EWC30-FM1}. 
			\item |Phase Error| $\leq$ 5.3° rms for \textbf{EWC30-FM2}. 
		\end{itemize}

		\end{itemize}
		
		\\\hline
		\textbf{Test Setup} &
		\begin{minipage}[t]{\linewidth}
		\begin{itemize}[nosep,after=\strut]
		\item \comEgse{}{} to DUT base-band electrical connections according to figure \ref{fig:ewc30_bb}
		\item General setup according to figure \ref{fig:data-setup1} and the following optional connections:
		\begin{itemize}
			%\item Frequency reference of PXA connected to 10 MHz reference of \fmn{}.
			\item RF input of PXA connected to \textbf{XB TP} of \fmr{}.
		\end{itemize}
	\end{itemize}
	\end{minipage}
	\\\hline
	\textbf{Duration} & 90 minutes\\\hline
	\textbf{Data sets required} &
	\begin{minipage}[t]{\linewidth}
			\begin{itemize}[nosep,after=\strut]
				\item \comEgse{} PXI nominal configuration file for EWC30 (\texttt{INIT\_FILE\_EWC30.ini}).
				\item Oscilloscope configuration files in \texttt{osc-config} folder
				\item Data file for modulation:
				\begin{itemize}[nosep,after=\strut]
					%\item Data-885840\_120s\_VCh01\_wPN.bin
					\item Data-4429200\_600s\_VCh01\_wPN.bin.
				\end{itemize}
				\item PXA configuration files in \texttt{\pxaTestFolderName} folder:
					\begin{itemize}[nosep,after=\strut]
						\item EWC30TX-FM1-VSA-v1.0.setx.
						\item EWC30TX-FM2-VSA-v1.0.setx.
						\item SB1FS-COM.csd.
					\end{itemize}
					\item Vector-0.9.4 script installed in \vmTesting.
		
				\end{itemize}
			\end{minipage}\\\hline
			\textbf{Prerequisites} & 
			\begin{minipage}[t]{\linewidth}
				\begin{itemize}[nosep,after=\strut]
					\item \preReqPro
					\item Hardware: The necessary items are shown in the table \ref{tab:req-items}.
				\end{itemize} 
			\end{minipage}\\\hline
		\end{tabular}
	\caption{Procedure \subprocid{} description. } \label{tb:ewctxfilter}
\end{table}


\begin{table}[H]
	\centering
	\begin{tabular}{|c|c|}
\hline
 Configuration\# & Filter Type and Advanced Cfg \\\hline
A10 (\refRD{comssrptP}) & SRRC filter, Roll-off = 0.5, Asym, Comp, LPF, HBF, LMS, DEAF \\\hline
B2 (\refRD{comssrptP}) & SRRC filter, Roll-off = 0.5, Asym, Comp, LPF, HBF, CMA \\\hline

\end{tabular}
	\caption{Filter configurations for Data demodulation.}
	\label{tab:data-advanced}
\end{table}



\newpage
\setcounter{Sec}{0}\setcounter{Step}{0}
\report{}{}{}
\begin{stepstable}{\subprocid{} \procname{}}
	\ExecutorRecord{}

	\SectionHeaderWithReset{Environmental temperature and humidity}
		\RegisterTemp
		\RegisterHumidity

	\SectionHeaderWithReset{PXA Connection and configuration}
		\ConnectRfCableToDwlTpPortOfCegse{XRF3.60}{\pasarSiAntes}
		\ConnectCableToDcBlockOnPXA{XRF3.60}{\pasarSiAntes}
		\ConfigurePxaInVSAMode
		%\ConnectPxaExternalFreqReferenceTo{the Power Splitter Data of \fmr{}}{REF1.01}{VSA}
		\LoadVSAConfigFile{EWC30TX-FM<X>-VSA-v1.0.setx}
		\LoadStateDefinition{SB1FS-COM.csd}

	\SectionHeaderWithReset{GS-GSE Preparation}
		\EnableControlInXbmaOf{\fmr{}}
		\EnableNOneInterfaceXBand{Note: Skip this step if \textbf{EWC30-FM2} is under test.}
	    \EnableNTwoInterfaceXBand{Note: Skip this step if \textbf{EWC30-FM1} is under test.}
		\SetXbmaAttenuationTo{0}


		\VerifyXBandDownConverterOne
	    \VerifyXBandDownConverterTwo
		\CreateScreenShotsFolderOnCortexHdrGse{\foldername{}}
		\ConfigureCortexHdrGse %{SB1GS-GSE-FM-N\_RF\_v<x.y>.mcs}{\versionNote{}}{CFG}
	 	\OpenMcsWindowsInCortexHdrGseForFilterTuninglTest
	 	\VerifyMatchedFilterParametersInCortexHdrGse


	 \SectionHeaderWithReset{EGSE Settings}
	 	%\SetCegseVariableAttenuatorTo{1 dB}{0 dB}
	 	\SetCegseVariableAttenuatorTo{10 dB}{0 dB}
		 \SectionHeaderWithReset{CEGSE SW Initialization}
		 \StartCegseSW{EWC30}{Nominal}{INIT\string_FILE\string_EWC30.ini}{\pasarSiAntes}
		 \SectionHeaderWithReset{DUT power on}
		 \VerifyDutAlarms{EWC30}{}
		 \TakeNoteDutTemperaturesXBand{}
		 \TurnOnOffVbusXBand{on}{\pasarSiAntes}
		 \VerifyRxSecondaryVoltageRFXBand
		 \VerifyRxSecondaryVoltageNUMXBand 
		 \VerifyTemperature{O\string_TX\string_TEMP1}{}
		 \LoadOscilloscopeConfigFile{EWC30-TX-RUN.set}{\pasarSiAntes}
		 \TakeNoteOfCurrentAndVoltageOf{TX}{CH1}{CH2}{28\ V}{< 282 mA}{}
		 \VerifyTxStatusXBand{Standby}

	 \SectionHeaderWithReset{Data transmission}
	 	%\VerifyTemperature{O\string_TX\string_TEMP1}{}
	 	\SendStoredFileWithDurationXBand{Data-4429200\_600s\_VCh01\_wPN.bin}{I\string_STBY\string_2\string_OPE\string_M}{10 minutes}

		\VerifyTxStatusXBand{Modulation}
		\VerifyClkLockedStatus{ON}
	    \VerifyRfOutputPowerTmXBand{$\approx$ 3.2}
		\TakeNoteOfCurrentAndVoltageOf{TX}{CH1}{CH2}{28\ V}{$\approx$ \ValueCurrentDUTTX A}{\notaCorrienteEstimada}


	 \SectionHeaderWithReset{VSA measurement}
	 	\StartRecordingRawWithPXA
	 	\SaveRecordedRawWithPXA{}
	 	\PlayVsaRecordingOnPXA
	 	\MeasureRfDataCharacteristicsOnVSA{-7.2 dBm $\pm1\ dB$ for \textbf{EWC30-FM1}\\ -7.3 dBm $\pm1\ dB$ for \textbf{EWC30-FM2} }
	 	\TakeScreenshotOnVSA{pause}{play}{HW}{1.png}

	\SectionHeaderWithReset{A10: SRRC roll-off=0.5, Asym, Comp, LPF, HBF, LMS, 
	 DEAF  filter meassurement}
	 	\RestartIFRInCortexHdr{1}
	 	\VerifyLockedInCortexHdr{1}
	 	\MeasureRfDataCharacteristicsOnCortex{}
	 	\TakeScreenShotInCortexHdrGse{\foldername{}}{a10.png}
      \VectorStar
	  \VectorVerify
	  \VectorStop


	\SectionHeaderWithReset{B2: SRRC roll-off=0.5, Asym, Comp, LPF, HBF, CMA 
	filter meassurement}
     \configureAdvancedParametersOnCortexHdrGse{2}
 	\VerifyLockedInCortexHdr{1}
	\MeasureRfDataCharacteristicsOnCortex
	\TakeScreenShotInCortexHdrGse{\foldername{}}{b2.png}
	\VectorStar
	\VectorVerify
	\VectorStop


     	\ChangeTxStatusToXBand{Standby}{M}{}
	 	\WaitUntilTxFinishedOnCegse		
	\SectionHeaderWithReset{Filter settings comparison.}
	\VectorGetEVM{option A10}
	\VectorGetEVM{option B2}


		\CompleteReportTable{Results for configurations filter of data demodulation}{}%{Results for configurations under test of Data demodulation}{}
     \verfifyTestSpecification{ For \textbf{A10} filter configuration: \\
		 - EVM < 6\% \\
		 - |Amplitude Error| < 0.5 dB rms  \\
		 - |Phase Error| < 5° rms for \textbf{EWC30-FM1}. \\
		 - |Phase Error| $\leq$ 5.3° rms for \textbf{EWC30-FM2}. 
	}
	 \SectionHeaderWithReset{DUT Power off}
	  	\TurnOnOffVbusXBand{off}{}

	 \SectionHeaderWithReset{CEGSE SW shutdown}
	 	\StopCegseSW{}

	 \SectionHeaderWithReset{Collect Evidences}

	\CopyPxaScreenShotsToCegseForEvidences%{pxa-screenshot and pxa-recording}
 	\CopyCegseLogsForEvicences{}{\guardarLogs}
    \CopyCortexHdrScreenShotsToCegseForEvidences{\foldername{}}
    \CopyFilesToCEGEFromMgmt{option A10}
	 \SectionHeaderWithReset{Final Steps}

	 	\CloseCortexConfigFileHdrGse	
		\SetChannelToReduntanSideOnXbmaOf{\fmr{}}{1}
		\SetChannelToReduntanSideOnXbmaOf{\fmr{}}{2}
		\DisconnectRfCableFromDwlTpOfCegse{XRF3.60}{}
		\DisconnectCableFromDcBlock{XRF3.60}{}	
		\RegisterTemp
		\RegisterHumidity


\end{stepstable}
\begin{longtable}{|p{17.0cm}|}
	\endfirsthead
	\endfoot
	\caption{\subprocid{} procedure.}
\end{longtable}


\begin{table}[H]
	\centering
	\begin{tabular}{|c|c|c|c|c|c|c|}
	\hline	
	Option Cfg \# & IF level & Eb/N0 & EVM (Vector Script) & Unb. Ratio(max) & Phase Error (max) \\\hline
	A10 &  & & & & \\\hline
	B2 &  & & & & \\\hline
	\end{tabular}
	\caption{Results for configurations filter of data demodulation.}
	\label{tab:tm-filter}
\end{table}